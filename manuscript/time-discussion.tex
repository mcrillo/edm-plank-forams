%%% DISCUSSION

% We found a positive correlation among species abundances through time and space, contrary to the community assembly patterns expected under strong interspecific competition. 
We found no evidence for interspecific competition structuring planktonic foraminifera communities. Thus there seems to be a mismatch between the processes inferred from macroevolutionary patterns over deep time and those inferred from ecological patterns observed in a shorter time scale. This means that either macroevolutionary processes are different than the current the ecological processes, or that interspecific competition might not be the mechanism underlying the patterns seen in the fossil record.

Scale problem: competition takes time, competition at ecological scales might generate diversification at evolutionary scales. 
Intraspecific competition promotes diversification theoretically (Doebeli, adaptive dynamics) and experimentally (Bailet et al. 2013 bacteria, increased richness, increased diversification; check david's intro as well) - what about interspecific competition? Niche partitioning / specialization first step to species diversification
Jablonski 2008: competition at the community scale might not have negative impact at the macroevolutionary scale.


A promising avenue of research, however, includes contrasting predictions from relevant theories within ecology and macroevolution, as well as embracing both abiotic and biotic proxies while modelling long-term evolutionary data \citep{voje2015role}. Biotic and abiotic affects population dynamics.

An important shortcoming of microfossils, for example compared to molluscs, is insufficient knowledge of their basic biology and natural history. Yet this current weakness is balanced by some distinctive strengths of the microfossils record, such as high abundance, large spatio-temporal coverage, and good taxonomic and temporal resolution (Yasuhara 2015).

Planktonic foraminifera (PF) unique fossil record has been used to test the diversity-dependent diversification model. PF speciation rates along the Cenozoic depend on the number of species present at the time, and decline as the number of species increases (DDD pattern), whereas extinction rate was driven largely by climate \cite{ezard2011interplay}. % 
More recently, \cite{ezard2016ecolet} showed that the diversification of the PF Cenozoic clade is regulated by competition among species, the strength of which varies through time as a function of environmental change. They invoke niche saturation and within-clade interspecific competition as the underlying mechanisms driving the DDD pattern without explicitly testing for it in the fossil record (because it is not possible actually).
% ezard2016ecolet:
% PF diversification dynamics is regulated by biotic "compensatory contest" competition, meaning that a constant number of successful individuals get the precise amount of resource they require, which is a fixed quantity (instead of the limiting resource being shared equally among all competing individuals - "scramble competition").
% temperature affects the per-lineage diversification rate, while both temperature and an environmental driver of sediment accumulation defines the carrying capacity
% contest competition constrains species richness by restricting niche availability, and that the number of macroevolutionary niches varies as a function of environmental changes.

A group may have more potential for coexistence among close relatives simply because that lineage has been present in that area for a longer amount of time.