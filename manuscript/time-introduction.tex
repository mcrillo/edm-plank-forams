
%%%%%%%%%%%%  INTRODUCTION  %%%%%%%%%%%% 


% Criticas Schluter e Pennell: ideias antiquadas,decada 80, devia testar com experimentos - JUSTIFICAR pq nao fiz isso:

% na intro: 
% (1) seria importante primeiro saber o que rola no mano-a-mano, pq no fundo nao sabemos o quanto de niche overlap essas especies tem, e no mano-a-mano seria um jeito de testar. O jeito ideal seria testar com experimentos de microcosmo, que a galera inclusive faz para protists,mas que com forams nao rola pq eles nao sao “cultivaveis” e isso tambem implica no nosso pequeno conhecimento da life history e ecology de plank forams
% (2) por isso testamos throughout a species range, que tambem eh importante pq a gente espera que sera nessa escala que uma especie afetaria a diversificacao da outra

% e ai na discussao posso falar de experimentos (chemiostatos) nos oceanos, em vez de no lab, que esse talvez seja o next step


%%% Macroevolution and community ecology - different scales

The interplay of ecological and evolutionary processes is central to our understanding of biodiversity patterns.
This connection is well accepted, but we still lack a powerful theory for how population-level processes scales up to clade-level dynamics, and vice versa \citep{jablonski2008biotic, weber2017evolution}.
One of the main reason this integration is still missing, is because community ecology and macroevolutionary patterns differ in temporal, spatial and taxonomic scales (Jablonski 2008, Weber 2017 and more refs). 
Community ecology, on the one hand, focusses on local and regional processes, happening along the life-span of individual organisms (which could range from hours to decades). The community is defined taxonomically as the species that co-occur in space and time, and therefore includes species from distantly related clades, from unicellular bacteria and protists to multicellular funghi, animal and plants. 
Macroevolution, on the other hand, focusses on processes that affect the diversification of species. The time scale is millions of years, scaled to the "life-span" of species, and the spatial scale encompasses species' ranges, which can be global. Moreover, macroevolution studies usually focus on monophyletic clades, and therefore species that share more recent evolutionary (and are taxonomically closer) when compared to species that live together in a community. 



%%% Eco-evolutionary feedbacks - the link between ecology and evolution

Although community ecology and macroevolution describe and seek to understand patterns at different scales, the processes generating these patterns are fundamentally linked through eco-evolutionary feedbacks. % focus on both biotic and abiotic factors!
Ecological interactions, which happen at the community scale, are known to shape species’ population dynamics (ref), alter natural selection (ref), and impact trait evolution (ref) and lineage diversification (DDD refs and others). % eco --> evo
Species evolution and diversification, in turn, respond to the changing environment and affect how species interact with the environment and with each other (refs, and more examples). % evo --> eco 
Even knowing that both abiotic and biotic factors play key role in community dynamics and clade diversification, the different scales of the observed patterns makes it hard to test hypothesis relevant to both fields.
% Attempts to integrates these two fields includes...
A promising avenue of research, however, includes contrasting predictions from accepted theories within macroevolution and ecology \citep{voje2015role}.
% A way forward to a mechanistic integration of these fields, however, is to contrast predictions from accepted theories within ecology and macroevolution \citep{voje2015role}.



%%% Diversity-dependent diversification theory --> mechanism: competition among species
% hard to directly test for biotic interactions in the fosil record
Within macroevolution theory, there is the hypothesis that higher levels of diversity tend to suppress rates of diversification (negative diversity-dependence diversification, DDD) (Sepkoski 1978; for a recent debate read: \citealt{harmon2015unbounded} and \citealt{rabosky2015limits}). 
DDD is used to describe both a pattern and a process within macroevolution \citep{rabosky2013diversity}. 
As a pattern, DDD describes a clade's bounded diversity trajectory through time (observed in the fossil record: Sepkoski 1978; Ezard 2011, 2016; and inferred from molecular filogenies: refs) and/or a negative relationship between standing taxonomic richness and rates of diversification across a clade's history (Foote 2018, Alroy, Quental).
As a process, DDD is usually thought to reflect competition among ecologically similar species and the filling of niche space (refs, but see Moen and Morlon TREE). This mechanism also explains species richness rebounds after mass extinctions events (refs), and has been tested by comparing predictions of the DDD and alternative diversification models, including models in which diversification is a function of temperature and rock sedimentation (Ezard 2016, Silvestro PNAS?, achar mais - ver filogenias molecular)(include Marshall and Quental 2016 somewhere).
Although the DDD patterns and models evoke competition among species, they do not explicitly test it, neither reveal how biotic interactions generated them \citep{jablonski2008biotic}. 
Explicitly testing for biotic interactions in the fossil record requires not only a high resolution fossil record (to get closer to temporal scales of community dynamics), but also an actual fossilized interaction proxy, both of which are so rare that until today only one study exists (Liow 2016 PRSB).
% What about works testing for niche saturation using trait evolution? 
% What about network studies, coevolution?


%%% Competition among species happens at the community level
% population-level traits that affect diversificaiton: abundance, range, connectivity...
To generate more mechanistic models of DDD, we need a better understanding of how and under what geographical and environmental circumstances species affect each others' diversification rates (i.e., speciation minus extinction rates) \citep{weber2017evolution}.  
Competitive interactions happen at the individual level. By competing with each other, individuals shape the dynamics of their populations and communities. Population dynamics, in turn, is directly linked to species' abundance and geographical range. And both, species' abundance and geographical range, influences species' diversification rates (Harnik 2011 PNAS, more refs). 
% Competition among species (i.e., interspecific) happens at the community level and is though to affect diversification of whole clades (DDD). 
At the community ecology scale, clade-wide interspecific competition is also hard to test, because you would need geographical and temporal data on population abundances of all (or most) species within a clade. 
Since it is hard to explicitly test for ecological interactions in the fossil record and observe clade-wide community dynamics, the empirical basis for an integration between community ecology and macroevolution theories has thus far been limited (e.g. molluscs; Jablonski et al., 2003, 2013).



%%% Microfossils as a model
Planktonic foraminifera (PF) represent a useful model system for such an integrative research (Yasuhara 2015 Bio Rev). PF are rhizarian protist that build a calcite shell, featuring the most complete fossil record of the Cenozoic Era currently known \citep{kucera2007chapter, ezard2011interplay}. Besides their simple morphology grants them a mature taxonomy, and the extant species are confirm by molecular studies (refs), although some morphospecies might encompass more than one genetic distinct type (cryptic species, refs). 
PF excellent fossil record has been used to test the DDD model. PF speciation rates along the Cenozoic depend on the number of species present at the time, suggesting interspecific competition affects speciation \cite{ezard2011interplay}. % 
More recently, \cite{ezard2016ecolet} showed that the diversification of the PF Cenozoic clade is regulated by competition among species, the strength of which varies through time as a function of environmental change. They invoke niche saturation and competition among species as the underlying mechanisms driving the DDD pattern ,but without explicitly testing for it in the fossil record.
Indeed it is yet not possible to test for PF ecological interactions in the fossil record, because of our lack of understanding of their population dynamics and ecology.
However, PF species live today as zooplankton in the marine pelagic environment, and their low diversity (46 extant morphospecies) allows a clade-wide study of their community dynamics.  
We can expect that, if competition is an important process of PF evolution, competition would also be an important ecological interaction among living PF species. 

%%%
%%% How to test for interspecific competition in today's communities? Communities = snapshot in time
%%%
% For individuals to compete for resources, there should be some level of niche overlap among individuals of the different taxa. 
%% Patterns that would support interspecific competition:
% Allopatry - competitive exclusion takes time, and ghost of past competition
% Correlation time-series

To test for interspecific competition in modern PF asseblages (obs: check use of population, community and assemblage), we analysed PF assemblage data spatially using 4177 assemblage counts around the world's oceans and temporally using 35 time-series collected globally from sediment traps. 
The essence of interspecific competition is that individuals of one species suffer a reduction in fecundity, growth or survivorship as a result of resource exploitation or interference by individuals of another species \citep{begon2006ecology}. This way, competition among individuals affects the population dynamics of the competing species, and these dynamics then influence the species' distributions and diversification. 
Two patters could emerge as ecological effects of interspecific competition: \textit{(i)} species may be eliminated from a habitat by competition from individuals of other species, resulting in a pattern of non-overlapping species’ ranges (i.e. allopatry); or, if competing species coexist, \textit{(ii)} individuals of at least one of them suffer reductions in abundance due to the presence of the other, leading to a pattern of negative correlation of abundances through time.
Sister species are morphogically similar (Supp Info - phylogenetic signal of shell size), thus we assume sister species pairs are ecologically more similar, and therefore compete more strongly, than randomly chosen species pairs. 


%%% Correlation is different than causality
EDM paragraph (correlation of time-series does not imply causality).


%%% Here

% taxonomically and geographically broad database
Here we analyse clade-wide population dynamics of living planktonic foraminifera species using spatial and temporal data to determine how populations of different species interact across their ranges and through ecological time. Given the PF diversity-dependent dynamics in their fossil record \citep{ezard2011interplay, ezard2016ecolet}, we expect interspecific competition to play a key role in structuring modern PF communities. 
% We investigate modern planktonic foraminifera communities and population dynamics to find patterns that would support interspecific competition. 




%%% Readings:

% Weber and Strauss 2016 AREES:On the one hand, the biological similarity and geographic origins of closely related species can promote their co-occurring in the same habitats. On the other hand, close relatives can ecologically, genetically, and reproductively interfere with one another owing to their morphological and ecological similarities. Community ecology generally excludes the deeper phylogenetic history of species pools, ignoring historical biogeography (see critiques by Cornell & Harrison (2014), Mittelbach et al. (2007), Ricklefs (2007)).
% Ecological processes that actually allow populations of species to have stable or positive population growth in sympatry. 
% Here, we define close relatives loosely as relatively recently diverged, morphologically, and/or ecologically similar species. 
% Coexistence requires nuanced consideration of spatial and temporal scale, migration/phenology, and microallopatry (Siepielski & McPeek 2010).
% Speciation and extinction can have prominent roles in shaping patterns of coexistence in close relatives.


% While there is empirical evidence in the fossil and neontological records of diversity dependence at play, we should note that diversity-dependence rates, although necessary, are not sufficient evidence to demonstrate the existence of a fixed carrying capacity [14,18,30,31]. In fact the results shown by Morlon et al. [32] based on several molecular phylogenies suggest that although rates of diversification consistently decrease over time, diversity for those same phylogenies are probably still expanding.

% The traditional diversity-dependent patterns are a negative relationship between standing taxonomic richness and diversification rate or speciation rate, or a positive relationship between richness and extinction rate (Alroy 1996; Wiens 2011; Cornell 2013). These traditional diversity-dependent signatures evoke macroevolutionary competition, but do not reveal how biotic interactions generated them (Jablonski 2008) nor whether a finite upper bound constrains species richness (Marshall & Quental 2016).

% Why just within a clade? Germain 2016, but silvestro quental 2016 PNAS
% Rachel M. Germain 2016: The diversity of ecological interactions on the Earth is the product of approximately 3.5 billion years of evolution, with ongoing extinctions matched by the continual divergence of populations and species. Signatures of this past evolution frequently emerge in the strength of the interactions among current-day species [1] in ways that have potential to further perpetuate divergence and the evolution of interaction strengths [2]. This dynamic feedback between the ecology and evolutionof organisms is a central theme in microevolutionary [3,4], macroevolutionary [5] and recent ecological perspectives [6–8], as it promises a more complete picture of the processes that generate and maintain biological diversity.

% A taxonomic group may limit species diversification or another group and therefore affect its species richness (Silvestro et al 2015), but the underlying mechanism of competition is still the same: individuals compete for resources. Thus the competitive process limiting diversification of whole clades also acts by affecting individual fitness, average population fitness and population size (contrary to line 53). You can think of different levels that competition affects: between individuals and suppresses access to resources, between populations and suppresses population sizes and between higher order taxa and suppresses species richness. However the underlying mechanism is always acting at the individual level. A population size can only be suppressed if their individuals have less access to resources and smaller reproductive success. And a species reduced diversification rate can only be suppressed via its populational sizes and average fitness. The results of these three levels of competition surely occur at different temporal, spatial and taxonomic scales. It takes longer for a species to go extinct than an individual to die. Species ranges are generally larger than individuals’ home ranges. The outcomes are expected to occur in different scales, however the underlying processes are all the same, at the same scale: competition among individuals.